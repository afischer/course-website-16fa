\documentclass[11pt,twoside]{article}
\usepackage{amsmath,amssymb,graphicx}
\usepackage{latexsym}

\newcommand{\coursenumber}{CIS 121}
\newcommand{\coursetitle}{Data Structures and Algorithms}
\newcommand{\docdate}{February 04, 2016}
\newcommand{\duedate}{February 04, 2016}
\newcommand{\doctitle}{Lecture Outline}
\newcommand{\student}{PUT YOUR NAME HERE}

\pagestyle{myheadings}
\markboth{\hfill\doctitle\hfill\docdate}{\docdate\hfill\doctitle\hfill}

\addtolength{\textwidth}{1.00in}
\addtolength{\textheight}{1.00in}
\addtolength{\evensidemargin}{-0.5in}
\addtolength{\oddsidemargin}{-0.5in}
\addtolength{\topmargin}{-.25in}
%\setlength{\footskip}{0pt}

\newcommand{\solution}{\bigskip\hrule\bigskip}
\newcommand{\problembreak}{\bigskip\hrule\bigskip}

\def\ni{\noindent}

\renewcommand{\theenumi}{\textbf{\Alph{enumi}}}

\newcommand{\comment}{$\rhd$\ \ }

\begin{document}

\thispagestyle{empty}

\begin{center}
\Large\bf\coursetitle\\[2pt]\doctitle\\ \large\docdate
% \\[2pt]\student       % Uncomment this line to insert your name
\end{center}
\vspace*{0.10in}
%{\noindent{\bf Given:}\ \ \docdate}
%{\hfill{\bf Due:}\ \ \duedate}

\hrule
~\\

\ni
\paragraph{Simplified Master Theorem.} Let $a\geq 1$, $b>1$ be
constants and let $T(n)$ be the recurrence 
\[
T(n) = aT\left (\frac{n}{b} \right ) + \Theta(n^k) 
\]
defined for $n\geq 0$ (we assume that $n$ is a power of $b$, though
this does not make a difference in asymptotic analysis). The base
case, $T(1)$ can be any constant value. Then \\
\textbf{Case 1: } if $a > b^k$, then $T(n)\in \Theta(n^{\log_b a})$.\\
\textbf{Case 2: } if $a = b^k$, then $T(n)\in \Theta(n^k\log_b n)$.\\
\textbf{Case 3: } if $a < b^k$, then $T(n)\in \Theta(n^k)$.

\paragraph{Example.} Using Master Theorem, prove bounds on $T(n)$ for
the following recurrences.
\begin{enumerate}
\item[a.] $T(n) = T(n/2) + n^2/2 + n$
\item[b.] $T(n) = 2T(n/4) + \sqrt{n}$
\item[c.] $T(n) = 3T(n/2) + (3/4)n$
\item[d.] $T(n) = 2T(n/4) + 5$
\end{enumerate}

\paragraph{Solution.} 
\begin{enumerate}
\item[a.] $a=1, b=2$, and $k=2$, thus case 3 applies and we get $T(n)
  = O(n^2)$.
\item[b.] $a=2, b=4$, and $k=1/2$, thus case 2 applies and we get $T(n)
  = O(\sqrt{n}\log n)$.
\item[c.] $a=3, b=2$, and $k=1$, thus case 1 applies and we get $T(n)
  = O(n^{\log_2 3})$.
\item[d.] $a=2, b=4$, and $k=0$, thus case 1 applies and we get $T(n)
  = O(n^{\log_4 2}) = O(\sqrt{n})$.
\end{enumerate}
There are recurrences for which none of the above cases apply. In that
case, we solve them using other techniques that we have learned. An
example of such a recurrence is $T(n) = T(n/4) + \lg n$.

\paragraph{Example.} 
Consider the following recurrence. We assume $n$ is a
power of 2. 
\[
T(n) = \left\{ \begin{array}{ll}
      2T(n/2) + n\lg n, &  n \geq 2 \\
      1, & \mbox{otherwise}
      \end{array} \right.
\]
Express $T(n)$ as $\Theta(f(n))$. 
\paragraph{Solution.} Note that the above recurrence cannot be solved using the Simplified Master Theorem. We use the method of expansion.
\begin{align*}
T(n) & =  2T(n/2) + n\lg n \\
& =  2^2T(n/2^2) + n\lg (n/2) + n\lg n \\
& = 2^3T(n/2^3) + n\lg (n/4) +n\lg (n/2) + n\lg n \\
&  \ldots \\
&  \ldots \\
& = 2^kT(n/2^k) + n\lg (n/2^{k-1}) +\ldots + n\lg (n/4) + n\lg (n/2) + n\lg n \\ 
& = 2^kT(n/2^k) + n\sum_{i=0}^{k-1}\lg (n/2^i)\\
& = 2^kT(n/2^k) + n\sum_{i=0}^{k-1}(\lg n - \lg 2^i)\\
& = 2^kT(n/2^k) + n\sum_{i=0}^{k-1}(\lg n - i)\\
& = 2^kT(n/2^k) + kn\lg n - nk(k-1)/2\\
& = 2^kT(n/2^k) + kn\lg n -nk^2/2 + nk/2\\
\end{align*}
\noindent
The recursion bottoms out when $n/2^k = 1$, i.e., $k=\lg n$. Thus, we get
\begin{align*}
T(n) & =  nT(1) + n\lg^2 n - n\lg^2 n/2 + n\lg n/2 \\
     & =  n + n\lg^2n/2 + n\lg n/2\\
     & =  \Theta(n\lg^2n)
\end{align*}


\end{document}
