\documentclass[11pt,twoside]{article}
\usepackage{amsmath,amssymb,graphicx}
\usepackage{latexsym}

\newcommand{\coursenumber}{CIS 121}
\newcommand{\coursetitle}{Data Structures and Algorithms}
\newcommand{\docdate}{January 26, 2016}
\newcommand{\duedate}{January 26, 2016}
\newcommand{\doctitle}{Running time, Divide and Conquer}
\newcommand{\student}{PUT YOUR NAME HERE}

\pagestyle{myheadings}
\markboth{\hfill\doctitle\hfill\docdate}{\docdate\hfill\doctitle\hfill}

\addtolength{\textwidth}{1.00in}
\addtolength{\textheight}{1.00in}
\addtolength{\evensidemargin}{-0.5in}
\addtolength{\oddsidemargin}{-0.5in}
\addtolength{\topmargin}{-.25in}
%\setlength{\footskip}{0pt}

\newcommand{\solution}{\bigskip\hrule\bigskip}
\newcommand{\problembreak}{\bigskip\hrule\bigskip}

\def\ni{\noindent}

\renewcommand{\theenumi}{\textbf{\Alph{enumi}}}

\newcommand{\ceil}[1]{\lceil#1\rceil}
\newcommand{\floor}[1]{\lfloor#1\rfloor}

\newcommand{\comment}{$\rhd$\ \ }

\begin{document}

\thispagestyle{empty}

\begin{center}
\Large\bf\coursetitle\\[2pt]\doctitle\\ \large\docdate
% \\[2pt]\student       % Uncomment this line to insert your name
\end{center}
\vspace*{0.10in}
%{\noindent{\bf Given:}\ \ \docdate}
%{\hfill{\bf Due:}\ \ \duedate}

\hrule

\paragraph{Example.}
Consider the following code fragment.
\begin{verbatim}
for (i = 0; i < n; i++)
    for (j = 0; j < i; j=j+10)
       print ("run time analysis")    
\end{verbatim}
Give a tight bound on the running time of this code fragment.

\paragraph{Solution.}
For each value of $i$, the body of the inner loop executes $i/10$
times. Thus the running time of the body of the outer loop is at most
$c(i/10)$, for some positive constant $c$. Hence the total running time of
the code fragment is given by
\[
\sum_{i=0}^{n-1}c\ceil{\frac{i}{10}} = \sum_{i=0}^{n-1}c\left (\frac{i}{10}+1\right ) = \frac{c(n-1)n}{20} +cn \leq 2cn^2 =O(n^2) 
\]
We will now show that $\sum_{i=0}^{n-1}c\ceil{\frac{i}{10}}=\Omega(n^2)$. Note that 
\[
\sum_{i=0}^{n-1}c\ceil{\frac{i}{10}}\geq \sum_{i=0}^{n-1}\frac{ci}{10} = c(n-1)n/20
\]
We want to
find positive constants $c'$ and $n_0$, such that for all $n\geq n_0$,  
\[
\frac{c(n-1)n}{20} \geq c'n^2
\]
This is equivalent to showing that $n(c-20c')\geq c$. This is true
when $c'=c/40$ and $n\geq 2$. Thus, the running time of the code
fragment is $\Omega(n^2)$.

\paragraph{Example.}
Consider the following code fragment.
\begin{verbatim}
i = n
while (i >= 10) do
  i = i/3
  for j = 1 to n do
    print ("Inner loop")
\end{verbatim}
What is an upper-bound on the running time of this code fragment? Is
there a matching lower-bound?

\paragraph{Solution.}
The running time of the body of the inner loop is $O(1)$. Thus the
running time of the inner loop is at most $c_1n$, for some positive constant
$c_1$. The body of the outer loop takes at most $c_2n$ time, for some
positive constant $c_2$ (note that the statement $i=i/3$ takes $O(1)$
time). Suppose the algorithm goes through $t$ iterations of the while
loop. At the end of the last iteration of the while loop, the value of $i$ is
$n/3^t$. We know that the code fragment surely finishes when
$n/3^{t}\leq 1$, solving which gives us $t\geq \log_3 n$. This means
that the number of iterations of the while loop is at most $O(\log
n)$. Thus the total running time is $O(n\log n)$.

We will now show that the running time is $\Omega(n^2)$. We will
lower-bound the number of iterations of the outer loop. Note that when
the value of $i$ is more than 10 (say, $3^3$), the outer loop has not
terminated. Solving $n/3^t\geq 3^3$, gives us that $\log_3n-3$ is a lower
bound on the number of iterations of the outer loop. For each
iteration of the outer loop, the inner loop runs $n$ times. Thus the
total running time is at least $cn(\log_3n-3)$, for some positive 
constant $c$. Note that $cn(\log_3 n-3)\geq c'n\log n$, when $c'=c/2$
and $n\geq 3^6$. Thus the running time is $\Omega(n\log n)$.

\paragraph{Example.} Consider the following code fragment.
\begin{verbatim}
for i = 0 to n do
  for j = n down to 0 do
    for k = 1 to j-i do
        print (k)
\end{verbatim}
What is an upper-bound on the running time of this algorithm? What is
the lower bound?
\paragraph{Solution.}
Note that for a fixed value of $i$ and $j$, the innermost loop goes
through $\max\{0,j-i\}\leq n$ times. Thus the running time of the
above code fragment is $O(n^3)$.

To find the lower bound on the running time, consider the values of
$i$, such that $0\leq i \leq n/4$ and values of $j$, such that $3n/4 \leq j
\leq n$. Note that for each of the $n^2/16$ different
combinations of $i$ and $j$, the innermost loop executes at least
$n/2$ times. Thus the running time is at least 
\[
(n^2/16)(n/2) = \Omega(n^3)
\]

\paragraph{Example.}
Consider the following code fragment.
\begin{verbatim}
for i = 1 to n do
  for j = 1 to i*i do
    for k = 1 to j do
        print (k)
\end{verbatim}
Give a tight-bound on the running time of this algorithm? We will
assume that $n$ is a power of 2.

\paragraph{Solution.} Note that the value of $j$ in the second
for-loop is upper bounded by $n^2$ and the value of $k$ in the
innermost loop is also bounded by $n^2$. Thus the outermost for-loop
iterates for $n$ times, the second for-loop iterates for at most $n^2$
times, and the innermost loop iterates for at most $n^2$ times. Thus
the running time of the code fragment is $O(n^5)$. 

We will now argue that the running time of the code fragment is
$\Omega(n^5)$. Consider the following code fragment.
\begin{verbatim}
for i = n/2 to n do
  for j = (n/4)*(n/4) to (n/2)*(n/2)  do
    for k = 1 to (n/4)*(n/4) do
        print (k)
\end{verbatim}
Note that the values of $i,j,k$ in the above code fragment form a
subset of the corresponding values in the code fragment in
question. Thus the running time of the new code fragment is a lower bound
on the running time of the code fragment in question. Thus the running
time of the code fragment in question is at least $n/2\cdot 3n^2/16 \cdot
n^2/16=\Omega(n^5)$.

Thus the running time of the code fragment in question is $\Theta(n^5)$. 

\paragraph{Example.}
Consider the following code fragment. We will assume that $n$ is a
power of 2.
\begin{verbatim}
for (i = 1; i <= n; i = 2*i) do
  for j = 1 to i do
        print (j)
\end{verbatim}
Give a tight-bound on the running time of this algorithm? 

\paragraph{Solution.} Observe that for $0\leq k\leq \lg n$, in the $k^{th}$ iteration of the
outer loop, the value of $i=2^{k}$. Thus the running time $T(n)$ of
the code fragment can be written as follows.
\begin{align*}
T(n) &= \sum_{k=0}^{\lg n}2^{k} \\
&= 2^{\lg n+1} -1 \\
&= 2n-1\\
&=\Theta(n)
\end{align*}
 
\ni
\textbf{Discussion:} Consider a problem $X$ with an algorithm $A$.
\begin{itemize}
\item
Algorithm $A$ runs in time $O(n^2)$. This means that the worst case
asymptotic running time of algorithm $A$ is upper-bounded by $n^2$.
Is this bound tight? That is,
is it possible that the run-time analysis of algorithm $A$ is
loose and that one can give a tighter upper-bound on the running time?  
\item
Algorithm $A$ runs in time $\Theta(n^2)$. This means that the bound
is tight, that is, a better (tighter) bound on the worst case
asymptotic running time for algorithm $A$ is not possible. 
\item
Problem $X$ takes time $O(n^2)$. This means that there is
an algorithm that solves problem $X$ on \textit{all} inputs in time
$O(n^2)$. 
\item
Problem $X$ takes $\Theta(n^{1.5})$. This means that there is
an algorithm to solve problem $X$ that takes time $O(n^{1.5})$ and
no algorithm can do better.
\end{itemize}

Consider the problem of
computing $2^n$ for any non-negative integer $n$. 
Below are four similar looking algorithms to
solve this problem. 

\begin{verbatim}
powerof2(n)
  if n = 0 
    return 1
  else
    return 2 * powerof2(n-1)

powerof2(n)
  if n = 0 
    return 1
  else
    return powerof2(n-1)+ powerof2(n-1)

powerof2(n)
  if n = 0 
    return 1
  else
    tmp = powerof2(n-1)
    return tmp + tmp

powerof2(n)
  if n = 0 
    return 1
  else
    tmp = powerof2(floor(n/2))
    if (n is even) then
      return tmp * tmp
    else
      return 2 * tmp * tmp 
\end{verbatim}

The recurrence for th first and the third method is 
$T(n) = T(n-1) + O(1)$. The recurrence for the second method is 
$T(n) = 2T(n-1)+O(1)$, and the recurrence for the last method is
$T(n)=T(n/2)+c$ (assuming that $n$ is a power of 2). 
In all cases the base case is $T(0) = 1$.  We
solve both these recurrences below. The recurrence for the first and
the third method can be solved as follows.\\

\begin{align*}
T(n) & =  T(n-1) + c \\
& =  T(n-2) + 2c \\
& =  T(n-3) + 3c\\
&  \ldots \\
&  \ldots \\
& =  T(n-k) + kc \\
\end{align*}
\noindent
The recursion bottoms out when $n-k=0$, i.e., $k=n$. Thus, we get
\begin{align*}
T(n) & =  T(0) + kc\\
     & =  1 + nc \\
     & =  \Theta(n)
\end{align*}

The recurrence for the second method can be solved as follows.
\begin{align*}
T(n) & =  2T(n-1) + c \\
& =  2^2T(n-2) + (2^0 + 2^1)c \\
& =  2^3T(n-3) + (2^0 + 2^1 + 2^2)c\\
&  \ldots \\
&  \ldots \\
& =  2^kT(n-k) + c\sum_{i=0}^{k-1}2^i \\
\end{align*}
\noindent
The recursion bottoms out when $n-k=0$, i.e., $k=n$. Thus, we get
\begin{align*}
T(n) & =  2^{n}T(0) + c\sum_{i=0}^{n-1}2^i \\ 
     & =  2^{n} + c(2^{n} -1) \\
     & =  \Theta(2^n)
\end{align*}

The recurrence for the fourth method can be solved as follows.
\begin{align*}
T(n) & =  T(n/2) + c \\
& =  T(n/2^2) + 2c \\
& =  T(n/2^3) + 3c \\
&  \ldots \\
&  \ldots \\
& =  T(n/2^k) + kc \\
\end{align*}
\noindent
The recursion bottoms out when $n/2^k < 1$, i.e., when $k> \lg n$. Thus, we get
\begin{align*}
T(n) & =  T(0) + c(\lg n + 1)\\
     & =  1  + \Theta(\lg n)\\
     & =  \Theta(\lg n)
\end{align*}

\end{document}
