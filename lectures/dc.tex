\documentclass[11pt,twoside]{article}
\usepackage{amsmath,amssymb,graphicx}
\usepackage{latexsym}

\newcommand{\coursenumber}{CIS 121}
\newcommand{\coursetitle}{Data Structures and Algorithms}
\newcommand{\docdate}{January 28, 2016}
\newcommand{\duedate}{January 28, 2016}
\newcommand{\doctitle}{Running time, Divide and Conquer}
\newcommand{\student}{PUT YOUR NAME HERE}

\pagestyle{myheadings}
\markboth{\hfill\doctitle\hfill\docdate}{\docdate\hfill\doctitle\hfill}

\addtolength{\textwidth}{1.00in}
\addtolength{\textheight}{1.00in}
\addtolength{\evensidemargin}{-0.5in}
\addtolength{\oddsidemargin}{-0.5in}
\addtolength{\topmargin}{-.25in}
%\setlength{\footskip}{0pt}

\newcommand{\solution}{\bigskip\hrule\bigskip}
\newcommand{\problembreak}{\bigskip\hrule\bigskip}

\def\ni{\noindent}

\renewcommand{\theenumi}{\textbf{\Alph{enumi}}}

\newcommand{\comment}{$\rhd$\ \ }

\begin{document}

\thispagestyle{empty}

\begin{center}
\Large\bf\coursetitle\\[2pt]\doctitle\\ \large\docdate
% \\[2pt]\student       % Uncomment this line to insert your name
\end{center}
\vspace*{0.10in}
%{\noindent{\bf Given:}\ \ \docdate}
%{\hfill{\bf Due:}\ \ \duedate}

\hrule

\subsection*{Linear Search and Binary Search} 
The input is an array $A$ of elements in any arbitrary order and a key $k$
and the objective is to output true, if $k$ is in $A$, false,
otherwise. Below is a recursive function to solve this problem.

\begin{verbatim}
LinearSearch (A[lo .. hi], k)
  if lo > hi then
    return False
  else
    return (A[hi] == k) or LinearSearch(A[lo..hi-1], k)
\end{verbatim}
The recurrence relation to express the running time of
\texttt{LinearSearch} is given by $T(n) = T(n-1)+c$, with the base
case being $T(0)=1$. We have already solved this recurrence and it
yields a running time of $T(n)=\Theta(n)$.

If the input array $A$ is already sorted, we can do significantly
better using \textit{binary search} as follows. 

\begin{verbatim}
BinarySearch (A[lo .. hi], k)
  if lo > hi then
     return False
  else 
     mid = floor(lo+hi/2)
     if A[mid] = k then
        return True
     else if A[mid] < k then
        return BinarySearch(A[mid+1 .. hi], k)
     else 
        return BinarySearch(A[lo .. mid-1], k) 
\end{verbatim}

The running
time of this method is given the recurrence $T(n)=T(n/2)+c$, with the
base case being $T(0)=1$. As we have seen before, this recurrence
yields a running time of $T(n)=\Theta(\log n)$.

\subsection*{Sorting}
Below is a recursive version of insertion sort that we studied a
couple of lectures ago.

\begin{verbatim}
InsertionSort(A[lo..hi])
  if lo = hi then
    return A
  else
    A' = InsertionSort(A[lo..hi-1])
    Insert(A', A[hi])  // insert element A[hi] into the sorted array A'   
\end{verbatim}
Note that the \texttt{Insert} function takes $\Theta(n)$ time for an
input array of size $n$. Thus the running time of Insertion sort is
given by the following recurrence.
\[
T(n) = \left\{ \begin{array}{ll}
      1, &  n=1 \\
      T(n-1) + n, &  n \geq 2 
      \end{array} \right.
\]
It is easy to see that this recurrence yields a running time of
$T(n)=\Theta(n^2)$.

To motivate the idea behind the next sorting algorithm (Merge Sort), let's rewrite
\texttt{InsertionSort} function as follows.  
\begin{verbatim}
InsertionSort(A[lo..hi])
  if lo = hi then
    return A
  else
    //Merge combines two sorted arrays into one sorted array   
    Merge(InsertionSort(A[lo..hi-1]), InsertionSort(A[hi..hi]))  
\end{verbatim}
The function \texttt{Merge} is as follows.
\begin{verbatim}
Merge(A[1..p], B[1..q])
  if p = 0 then 
    return B
  if q = 0 then
    return A
  if A[1] <= B[1] then
    return prepend(A[1], Merge(A[2..p], B[1..q])) 
  else
    return prepend(B[1], Merge(A[1..p], B[2..q]))
\end{verbatim}
Note that the running time of \texttt{Merge} is $O(p+q)$. 
The second recursive call to \texttt{InsertionSort} takes
$O(1)$ time and hence the running time of \texttt{InsertionSort} still is
$\Theta(n^2)$. 

Observe that in \texttt{InsertionSort} the input array $A$ is partitioned
into two arrays, one of size $|A|-1$ and another of size 1. In Merge
Sort, we partition the input array of size $n$ in two equal halves (assuming $n$ is
a power of 2). Below is the function.

\begin{verbatim}
MergeSort(A[1..n])
if n = 1 then
  return A
else
  return Merge(MergeSort(A[1..n/2], MergeSort(A[n/2+1..n]))
\end{verbatim}
The running time of \texttt{MergeSort} is given by the following
recurrence.
\[
T(n) = \left\{ \begin{array}{ll}
      1, & n=1 \\
     2T(n/2) + cn, &  n \geq 2 
            \end{array} \right.
\]

Below are some facts on logarithms that you may find useful.
\begin{enumerate}
\item[i.]
$\log_a b = \frac{1}{\log_b a}$
\item[ii]
$\log_a b = \frac{\log_c b}{\log_c a}$
\item[iii]
$a^{\log_a b} = b$
\item[iv] $b^{\log_a x} = x^{\log_a b}$ 
\end{enumerate}

\ni
We can also solve recurrences by guessing the overall form of the
solution and then figure out the constants as we proceed with the
proof. Below are some examples.

\paragraph{Example.} Consider the following recurrence for the \texttt{MergeSort} algorithm.
\[
T(n) = \left\{ \begin{array}{ll}
        1, & n=1\\
        2T(n/2) + n, &  n \geq 2 

      \end{array} \right.
\]
Prove that $T(n)=O(n\lg n)$.

\paragraph{Solution.} We will first prove the claim by expanding the recurrence as follows.
\begin{align*}
T(n) & =  2T(n/2) + n \\
& =  2^2T(n/2^2) + 2n \\
& =  2^3T(n/2^3) + 3n\\
&  \ldots \\
&  \ldots \\
& =  2^kT(n/2^k) + kn \\
\end{align*}
\noindent
The recursion bottoms out when $n/2^k=1$, i.e., $k=\lg n$. Thus, we get
\begin{align*}
T(n) & =  2^{\lg n}T(1) + n\lg n\\
     & =  \Theta(n\log n)\\
\end{align*}

\ni
We will now prove that $T(n)=O(n\lg n)$
by using strong induction on $n$. We will show that for 
some constant $c$, whose value we will determine later, $T(n)
\leq cn\lg n$, for all $n\geq 2$. \\
\underline{Induction Hypothesis:} Assume that the claim is true when
$n=j$, for all $j$ such that  $2\leq j\leq k$. In other words,
$T(j)\leq cj\lg j$. \\
\underline{Base Case:} $n=2$. The left
hand side is given by $T(2) = 2T(1) + 2 = 4$ and the right hand side
is $2c$. Thus the claim is true for the base case when $c\geq 2$.\\
\underline{Induction Step:} We want to show that for $k\geq 2$, $T(k+1)\leq
c(k+1)\lg (k+1)$. We have
\begin{align*}
T(k+1) & = 2T\left (\frac{k+1}{2}\right ) + (k+1) \\
       & \leq 2c\left (\left (\frac{k+1}{2}\right )\lg \left
(\frac{k+1}{2}\right )\right ) + (k+1) \\
       & = c(k+1)(\lg (k+1) - \lg 2)  + (k+1) \\
       & = c(k+1)\lg (k+1) - (c-1)(k+1)  \\
       & \leq c(k+1)\lg (k+1) ~~~~~~~ (\mbox{since } c\geq 2)
\end{align*}

\paragraph{Example.} Consider the following recurrence.
\[
T(n) = \left\{ \begin{array}{ll}
      1, & n = 1\\
      2T(n/2) + n^2, &  n \geq 2 
      \end{array} \right.
\]
Prove that $T(n)=\Theta(n^2)$.  
\paragraph{Solution.} Clearly, $T(n)=\Omega(n^2)$ (because of the
$n^2$ term in the recurrence). To prove that $T(n)=O(n^2)$, we will
show using strong induction that for 
some constant $c$, whose value we will determine later, $T(n)
\leq cn^2$, for all $n\geq 1$. \\
\underline{Induction Hypothesis:} Assume that the claim is true when
$n=j$, for all $j$ such that  $1\leq j\leq k$. In other words,
$T(j)\leq cj^2$. \\
\underline{Base Case:} $n=1$. The claim is clearly true as the left
hand side and the right hand side, both equal 1. \\
\underline{Induction Step:} We want to show that $T(k+1)\leq
c(k+1)^2$. We have
\begin{align*}
T(k+1) & = 2T\left (\frac{k+1}{2}\right ) + (k+1)^2 \\
       & \leq 2c\left (\frac{k+1}{2}\right )^2 + (k+1)^2 \\
       & = \left (\frac{c}{2} + 1\right )(k+1)^2 
\end{align*}
We want the right hand side to be at most $c(k+1)^2$. This means that we
want $c/2 + 1 \leq c$, which holds when $c\geq 2$. Thus we have shown
that $T(n)\leq 2n^2$, for all $n\geq 1$, and hence $T(n)=O(n^2)$.

\end{document}
